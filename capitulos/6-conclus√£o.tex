\chapter{Conclusão}

Este trabalho apresentou o desenvolvimento e a aplicação de um sistema inovador, o Emotion Analytics, para a análise de emoções em tempo real em aplicações web. O objetivo principal foi fornecer aos desenvolvedores e pesquisadores uma ferramenta eficaz para compreender a experiência emocional dos usuários durante a utilização de software, permitindo a identificação de possíveis problemas, melhorias e otimizações no design das aplicações.

A pesquisa foi fundamentada na relevância e na importância que a análise e avaliação das emoções dos usuários desempenham no campo da Interface Humano-Computador (IHC). A compreensão das emoções dos usuários durante a interação com sistemas de software é crucial para o desenvolvimento de interfaces mais eficientes e satisfatórias. Ao analisar e compreender as emoções dos usuários, é possível identificar pontos fortes e fracos das interfaces, permitindo melhorias no design e na usabilidade dos sistemas.

O sistema Emotion Analytics foi desenvolvido utilizando a metodologia FACS (Facial Action Coding System) de Paul Ekman \cite{5} como base para a identificação e classificação das expressões faciais relacionadas às emoções. A escolha desse mecanismo deve-se à sua ampla utilização e reconhecimento na área, permitindo uma análise precisa e objetiva das emoções dos usuários durante a interação com o software.

Os experimentos realizados com o sistema Emotion Analytics mostram sua eficácia e precisão na análise de emoções. Os usuários interagiram com diferentes tipos de aplicações web, incluindo um jogo, um site de notícias e uma plataforma de vídeos, enquanto o sistema registrava suas emoções através da webcam do computador. Os dados coletados forneceram insights valiosos sobre a experiência emocional dos usuários, que podem ser usados para melhorar a interface e a usabilidade dos sistemas testados.

Este trabalho não só contribuiu para o avanço da área de Interação Humano-Computador (IHC), fornecendo uma ferramenta inovadora para a análise e compreensão da experiência emocional dos usuários, mas também alcançou reconhecimento acadêmico. A pesquisa resultou na publicação de um artigo no "21º Simpósio de Iniciação Científica e Tecnológica da Faculdade de Tecnologia de São Paulo" \cite{25}, destacando a relevância e o impacto deste estudo.

Além disso, a disponibilidade do Emotion Analytics como projeto de código aberto e a publicação do código-fonte no GitHub \cite{22}\cite{23} promovem a transparência, incentivam a colaboração e facilitam a replicação da pesquisa por outros interessados na área.

Em conclusão, o trabalho realizado neste TCC monstram a importância e a eficácia da análise de emoções na interação humano-computador. O sistema Emotion Analytics provou ser uma ferramenta valiosa para a compreensão da experiência do usuário, fornecendo percepções que podem ser usadas para melhorar a interface e a usabilidade dos sistemas. A publicação do artigo no Simpósio de Iniciação Científica e Tecnológica da Faculdade de Tecnologia de São Paulo reforça a relevância deste trabalho. Espera-se que este trabalho inspire pesquisas futuras na área e contribua para o desenvolvimento de aplicações web mais eficientes, intuitivas e emocionalmente engajadoras.
