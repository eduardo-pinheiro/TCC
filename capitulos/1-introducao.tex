\chapter{Introdução}

\section{Sobre a área de atuação}

A área de Interação Humano-Computador (IHC) é um campo interdisciplinar que se concentra no estudo da interação entre seres humanos e sistemas computacionais, visando projetar e desenvolver interfaces que sejam eficientes, eficazes e proporcionem uma experiência positiva aos usuários \cite{1}.

A área de IHC é de extrema importância na atualidade, à medida que a tecnologia desempenha um papel cada vez mais central em nossa sociedade. Com a crescente dependência de sistemas computacionais em diversos aspectos de nossas vidas, desde dispositivos móveis e aplicações web até sistemas complexos de automação, é crucial garantir que essas interações sejam intuitivas, acessíveis e adequadas às necessidades dos usuários.

A interação eficaz entre humanos e computadores é essencial para melhorar a usabilidade, a produtividade e a satisfação dos usuários. Ao projetar interfaces que sejam intuitivas e fáceis de usar, podemos reduzir a curva de aprendizado e minimizar erros, proporcionando aos usuários uma experiência agradável e eficiente \cite{2}. Além disso, a área de IHC também está preocupada em considerar aspectos emocionais e sociais da interação, buscando promover engajamento, confiança e bem-estar dos usuários.

Diversos trabalhos semelhantes já foram desenvolvidos na área de IHC, abrangendo desde estudos teóricos até aplicações práticas. Entre eles, destaca-se o trabalho intitulado "O Emocard na avaliação da interação do usuário no WebGD", realizado por Angela R. B. Flores \cite{3}. Nessa pesquisa, foi proposto o uso do Emocard, uma ferramenta baseada em análise de emoções, para avaliar a interação dos usuários com o sistema WebGD.

O estudo conduzido por Flores utilizou o Emocard para capturar as emoções dos usuários durante a interação com o WebGD, um ambiente de desenvolvimento web. Por meio da análise das expressões faciais e outras reações fisiológicas dos participantes, o Emocard permitiu identificar e mensurar as emoções experimentadas pelos usuários durante a utilização do sistema.

Os resultados obtidos no trabalho mostraram que o Emocard foi eficaz na detecção de emoções, fornecendo informações valiosas sobre a experiência dos usuários com o WebGD. Com base nesses dados, foi possível compreender como as emoções influenciaram a interação e identificar aspectos do sistema que causaram respostas emocionais positivas ou negativas nos usuários.

Ao considerar o trabalho de Flores, observamos a relevância do uso de métodos e ferramentas baseados em análise de emoções para avaliar a interação humano-computador. Essa abordagem permite uma compreensão mais aprofundada das percepções e sentimentos dos usuários durante o uso de sistemas, auxiliando no aprimoramento da usabilidade e na identificação de possíveis problemas ou oportunidades de melhoria.

O estudo realizado no artigo aponta uma abordagem semelhante a este trabalho no que se refere à utilização de emoções para análise e avaliação de software. No entanto, existem diferenças significativas entre a pesquisa estudada e esta pesquisa em relação à natureza e identificação das emoções dos usuários. No estudo anterior, as emoções relatadas pelos usuários foram classificadas como "aflição, descontentamento, depressão, sonolência, despertar, relaxamento, prazer e excitação", enquanto nesta pesquisa utilizamos as emoções "felicidade, nojo, raiva, medo, desprezo e tristeza" \cite{4}.

Além disso, os métodos de coleta das emoções também diferem entre os estudos. Na pesquisa anterior, os usuários eram solicitados a assinalar as emoções sentidas ao final da experiência de uso, enquanto nesta pesquisa as emoções são coletadas durante a própria experiência de uso por meio de algoritmos baseados na metodologia FACS de Paul Ekman \cite{5}.

Foi realizada uma extensa pesquisa bibliográfica visando encontrar estudos que representassem a experiência de uso de interfaces utilizando a metodologia FACS de Paul Ekman \cite{5}. No entanto, até o momento, não encontramos nenhum estudo que explore especificamente essas características, o que torna esta pesquisa inovadora e pioneira nesse sentido.

Essa abordagem original abre novas possibilidades para a compreensão da experiência do usuário em relação às interfaces de software, fornecendo percepções valiosas sobre as emoções e reações dos usuários durante a interação.

\section{Justificativa}

A justificativa para escrever sobre o assunto abordado neste trabalho é fundamentada na relevância e na importância que a análise e avaliação das emoções dos usuários desempenham no campo da Interface Humano-Computador (IHC). A seguir, são apresentados os principais motivos que justificam a escrita deste trabalho.

Primeiramente, a compreensão das emoções dos usuários durante a interação com sistemas de software é crucial para o desenvolvimento de interfaces mais eficientes e satisfatórias. As emoções exercem um papel significativo na forma como os usuários percebem, reagem e se envolvem com as aplicações, influenciando diretamente a experiência do usuário. Ao analisar e compreender as emoções dos usuários, é possível identificar pontos fortes e fracos das interfaces, permitindo melhorias no design e na usabilidade dos sistemas \cite{6}.

Além disso, a análise das emoções dos usuários fornece percepções valiosas sobre suas preferências, necessidades e expectativas. Ao compreender as emoções específicas evocadas pelas interfaces, os desenvolvedores podem personalizar as interações, adaptando-as às preferências individuais dos usuários. Essa abordagem personalizada contribui para a satisfação do usuário, aumentando a eficiência e a eficácia dos sistemas \cite{7}.

Outro aspecto relevante é a importância da análise das emoções dos usuários no contexto de aplicações específicas, como jogos, sites de notícias, redes sociais, entre outros. Cada tipo de aplicação possui características e objetivos distintos, e compreender as emoções dos usuários nesses contextos específicos permite desenvolver interfaces mais adequadas e atraentes. A análise das emoções dos usuários em diferentes domínios de aplicação contribui para a criação de experiências mais envolventes e satisfatórias.

Além disso, a utilização de metodologias e ferramentas computacionais para a análise das emoções apresenta uma abordagem inovadora e promissora no campo da IHC. A aplicação de algoritmos e técnicas de processamento de emoções permite coletar e interpretar grandes volumes de dados emocionais de maneira precisa e eficiente, fornecendo uma base sólida para a tomada de decisões no design e na avaliação de interfaces \cite{8}.

Por fim, a escassez de estudos relacionados à análise e avaliação das emoções dos usuários no contexto específico abordado neste trabalho também justifica a relevância deste estudo. Ao preencher essa lacuna na literatura acadêmica, este trabalho contribui para o avanço do conhecimento na área, fornecendo visões valiosas e abrindo novas perspectivas de pesquisa.

Dessa forma, considerando a importância da compreensão das emoções dos usuários na IHC, a relevância da análise em diferentes domínios de aplicação, a aplicação de metodologias computacionais e a necessidade de preencher a lacuna existente. Torna-se fundamental escrever sobre o assunto abordado neste trabalho, visando contribuir para o avanço da área e proporcionar benefícios tangíveis para o desenvolvimento de interfaces mais eficientes, personalizadas e envolventes.

\section {Objetivos}

Neste TCC, temos como objetivo principal a construção de um sistema de análise de emoções em tempo real em aplicações web. O sistema, denominado Emotion Analytics, será desenvolvido com o propósito de capturar, identificar e analisar as emoções dos usuários durante a interação com as aplicações.

O principal objetivo é fornecer aos desenvolvedores e pesquisadores uma ferramenta eficaz para compreender a experiência emocional dos usuários durante a utilização de software. Isso permitirá a identificação de possíveis problemas, melhorias e otimizações no design das aplicações, além de possibilitar uma análise mais precisa do impacto emocional das interfaces.

Os objetivos específicos deste trabalho são:

\begin{enumerate}
  \item Desenvolver um software capaz de capturar e identificar as emoções dos usuários em tempo real durante a interação com aplicações web.
  \item Utilizar a metodologia FACS (Facial Action Coding System) de Paul Ekman como base para a identificação e classificação das expressões faciais relacionadas às emoções \cite{5}.
  \item Integrar o sistema Emotion Analytics a diferentes tipos de aplicações web, permitindo a coleta e análise de dados emocionais em diversas áreas de atuação.
  \item Validar o sistema por meio de experimentos e estudos de caso, a fim de verificar sua eficácia e precisão na análise de emoções.
\end{enumerate}
  
Ao alcançar esses objetivos, espera-se que o sistema Emotion Analytics contribua significativamente para o desenvolvimento de aplicações web mais eficientes, intuitivas e emocionalmente engajadoras. Além disso, pretende-se fornecer subsídios teóricos e práticos para pesquisadores e profissionais da área, estimulando o avanço e a adoção de técnicas de análise de emoções no campo da IHC.

\section{Conteúdo}

Neste capítulo "Introdução" (Capítulo 1), foi contextualizada a pesquisa na área de Interação Humano-Computador (IHC) destacando a importância da análise de emoções na experiência do usuário. O "Referencial teórico" (Capítulo 2) fornece a base teórica, explorando a relevância das emoções na IHC e os métodos para identificação de emoções. A "Metodologia" (Capítulo 3) detalha os experimentos realizados com três sites diferentes, utilizando o sistema Emotion Analytics para registrar as emoções dos usuários. O "Software construído" (Capítulo 4) descreve o desenvolvimento do sistema Emotion Analytics, incluindo sua arquitetura, algoritmos e funcionalidades. Os "Resultados e análises" (Capítulo 5) apresentam os achados dos experimentos e discutem a eficácia do sistema na detecção de emoções. A "Conclusão" (Capítulo 6) resume os principais resultados da pesquisa e suas implicações para a área de IHC. Os "Agradecimentos" (Capítulo 7) expressam gratidão a todos que contribuíram para a realização do trabalho. Por fim, a "Bibliografia" (Capítulo 8) lista todas as referências utilizadas na pesquisa.
