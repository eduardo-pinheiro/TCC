%====================================================================
% Introdução: Escreva logo após o \chapter{Introdução} a texto 
% de seu trabalho referente a introdução.
%====================================================================
\chapter{Resultados e análises}

\lipsum[1-2]

\nomenclature[A]{MVF}{Método dos Volumes Finitos}
\nomenclature[S]{$a$}{aceleração}

De acordo com Burden \cite{burden} e Leithold \cite{leithold}. 
Temos ainda os trabalhos de Gooch e Daniel \cite{gooch,daniel}, 
Santana \cite{santana}. Usando o \emph{gmsh} \cite{gmsh} e 
referências na internet \cite{wiki:quadrature}.

\section{Primeira Seção}

\lipsum[3-4]

\nomenclature[A]{MEF}{Método dos Elementos Finitos}
\nomenclature[S]{$\rho$}{densidade}

\section{Segunda Seção}

\lipsum[5-6]

\nomenclature[A]{MDF}{Método das Diferenças Finitas}
\nomenclature[S]{$\omega$}{frequência}
\nomenclature[A]{MDFA}{Método das Diferenças Finitas Avançado}

\section{Primeira Seção}

\lipsum[3-4]

\nomenclature[A]{MEF}{Método dos Elementos Finitos}
\nomenclature[S]{$\rho$}{densidade}

\section{Segunda Seção}

\lipsum[5-6]

\nomenclature[A]{MDF}{Método das Diferenças Finitas}
\nomenclature[S]{$\omega$}{frequência}

\section{Terceira Seção}

\lipsum{10}