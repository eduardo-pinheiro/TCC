\chapter{Referencial teórico}

Neste capítulo, apresentaremos o referencial teórico que sustenta a pesquisa, abordando as referências bibliográficas utilizadas, as emoções e sua relevância na interação humano-computador, os mecanismos existentes para identificação de emoções, o mecanismo escolhido e as linguagens de desenvolvimento utilizadas no software.

As referências bibliográficas que fundamentam este trabalho incluem estudos relevantes na área de Interação Humano-Computador (IHC) e emoções. Dentre elas, destacam-se os trabalhos de Padovani \cite{1}, que aborda a Avaliação Ergonômica de Sistemas de Navegação em Hipertextos Fechados, Nielsen \cite{2}, com seu trabalho sobre usabilidade, e Norman \cite{6}\cite{7}, que explora os princípios do design centrado no usuário no livro "The Design of Everyday Things". Além disso, são considerados os estudos de Ekman \cite{4}\cite{5}, renomado pesquisador no campo das emoções, como "Basic Emotions" e "Emotions Revealed", e Flores \cite{3}, que propõe o uso do Emocard na avaliação da interação do usuário no WebGD.

Emoções desempenham um papel significativo na interação humano-computador, influenciando a experiência do usuário e seu engajamento com o software. Capturar as emoções durante a utilização de um sistema é fundamental para compreender o envolvimento emocional dos usuários, sua satisfação e nível de engajamento. Essa captura proporciona visões valiosas para o aprimoramento da usabilidade e da experiência do usuário, permitindo o desenvolvimento de interfaces mais adequadas e eficientes.

Existem diversos mecanismos de identificação de emoções, que variam desde abordagens tradicionais, como questionários e entrevistas, até abordagens mais tecnológicas e automatizadas. Entre os mecanismos existentes, destacam-se a análise facial computacional, a análise de expressões vocais, a detecção de padrões fisiológicos e a análise de interações comportamentais. Cada um desses mecanismos possui suas vantagens e desafios, e a escolha do mais adequado depende das características do sistema e dos objetivos da pesquisa.

Neste trabalho, optamos por utilizar o mecanismo baseado na metodologia FACS (Facial Action Coding System) desenvolvida por Paul Ekman \cite{5}. Essa metodologia utiliza algoritmos de análise facial computacional para identificar e classificar as expressões faciais associadas a diferentes emoções. A escolha desse mecanismo deve-se à sua ampla utilização e reconhecimento na área, permitindo uma análise precisa e objetiva das emoções dos usuários durante a interação com o software.

O mecanismo utilizado recebe o nome de Affectiva \cite{8} e tem como objetivo analisar imagens de rostos em larga escala, convertendo as imagens das faces coletadas em dados que apresentam as emoções desses rostos, avaliando suas micro expressões faciais \cite{4}. Esse framework foi desenvolvido por uma equipe de pesquisadores do MIT \cite{8} sendo o escolhido para a pesquisa devido à sua alta precisão e confiabilidade \cite{9}\cite{10}.

No que diz respeito às linguagens de desenvolvimento utilizadas no software, foram adotadas tecnologias e linguagens apropriadas para a implementação das funcionalidades propostas. Entre as linguagens comumente utilizadas estão HTML \cite{11}, CSS \cite{12}, JavaScript \cite{13}, Node.js \cite{14}, ReactJS \cite{15}, Redux \cite{16}, Ruby \cite{17}, entre outras. A escolha dessas linguagens baseou-se nas necessidades e requisitos específicos do projeto, visando a criação de uma aplicação robusta e interativa.

Dessa forma, o referencial teórico apresentado neste capítulo fundamenta a pesquisa. Fornecendo embasamento conceitual e acadêmico sobre as referências bibliográficas utilizadas, a importância da captura de emoções na interação humano-computador, os mecanismos de identificação de emoções existentes, o mecanismo escolhido baseado na metodologia FACS de Ekman \cite{5} e as linguagens de desenvolvimento empregadas no software.
