%====================================================================
\documentclass[12pt,openright,oneside]{book}
%====================================================================
% Pacote de estilo que tem por finalidade carregar pacotes do LaTeX
% bem como definir configurações do texto da monografia. 
%====================================================================
\usepackage{estilo}
%====================================================================
% Inserção do arquivo com a lista de abreviaturas e símbolos.
%====================================================================
\makenomenclature
%====================================================================
% Documento Principal
%====================================================================
\begin{document}
\tolerance=5000
%====================================================================
% Para que as linhas do texto sejam numeradas com a finalidade 
% de facilitar a correção pelo orientador, basta descomentar o
% o comando abaixo. Esse comando faz com que o arquivo pdf gerado 
% após a compilação sem erros tenham as linhas da monografia sejam
% enumeradas. Com as linhas numeradas o orientador poderá localizar 
% e descrever melhor para os orientandos os locais do texto que 
% exigem correção.
%====================================================================
%\linenumbers
%====================================================================
% O comando \capa tem 6 argumentos de entrada os quais são 
% descritos abaixo. Tem por função gerar a capa da monografia. 
%
% \capa
% {título da monografia}
% {nome do orientando}
% {ano de defesa da monografia}
% {nome do orientador}
% {nome do segundo membro da banca}
% {nome do terceiro membro da banca}
%
% Observações: 
%
%  1- Escrever o título do trabalho em letras maíusculas.
%       Exemplo: ANÁLISE DE REGRESSÃO EM PROBLEMAS DE FÍSICA
%  2- Nome do aluno com iniciais em letras maiúsculas.
%       Exemplo: Alberto Barbosa da Cunha
%  3- Ano da defesa: 
%       Exemplo: 2018
%  4- Nome do Orientador e demais membros da banca de defesa: 
%     Escrever o nome apenas com as iniciais em letras maiúsculas 
%     como no nome do aluno.
%       Exemplo: João Carlos Silva de Almeida    
%====================================================================
\capa
{SISTEMA PARA AVALIAÇÃO DE INTERFACES A
PARTIR DAS EMOÇÕES DOS USUÁRIOS}
{Eduardo Felipe Santos Pinheiro}
{2023}
{Marcelo Duduchi Feitosa}
{Nome completo do segundo membro da banca de defesa}
{Nome completo do terceiro membro da banca de defesa}
%====================================================================
% Agradecimentos: Os agradecimentos deverão ser escritos dentro do 
% arquivo agradecimentos.tex. 
%====================================================================     
% Observação: O arquivo agradecimentos.tex já existe dentro da 
% pasta monografia. O que o aluno precisa fazer é abrí-lo 
% e preenchê-lo com o texto dos agradecimentos.  
%====================================================================
%====================================================================
% Agradecimentos: Escreva no ambiente abaixo os agradecimentos.
%  
%  Exemplo:
%  
%  \begin{agradecimentos}
%     Agradeço a toda minha família e a todos os meus amigos 
%     pelo apoio nessa importante fase do meu processo de formação
%     profissional.
%  \end{agradecimentos}
%====================================================================

\begin{agradecimentos}
  Primeiramente, gostaria de expressar minha profunda gratidão ao CNPQ pela oportunidade de participar do Programa Institucional de Bolsas de Iniciação Científica (PBIC). A experiência e o conhecimento adquiridos durante a minha participação no programa foram fundamentais para a realização deste trabalho. Agradeço pela confiança depositada em mim e pelo suporte contínuo que me permitiu desenvolver e aprimorar minhas habilidades de pesquisa.
  \\

  Em segundo lugar, gostaria de agradecer ao meu orientador, o Professor Marcelo Duduchi. Sua orientação, paciência e conhecimento foram inestimáveis para a realização deste trabalho. Agradeço por ter me desafiado a pensar criticamente, a questionar e a buscar soluções inovadoras para os desafios encontrados durante o desenvolvimento do projeto. Sua orientação e apoio foram fundamentais para o sucesso deste trabalho.
  \\

  Por fim, gostaria de expressar minha gratidão à Faculdade de Tecnologia de São Paulo (FATEC-SP). Agradeço a todos os professores e colegas que contribuíram para a minha formação acadêmica e pessoal. A experiência adquirida durante o curso foi fundamental para a realização deste trabalho e para o meu desenvolvimento como profissional na área de tecnologia da informação.
  \\
  
  Este trabalho é o resultado de muitas horas de estudo, pesquisa e dedicação, e eu sou grato por todas as pessoas e instituições que me apoiaram ao longo desta jornada. Agradeço a todos que contribuíram, direta ou indiretamente, para a realização deste trabalho.
\end{agradecimentos}

%====================================================================
% Resumo em português: O resumo em português deverá ser escrito 
% dentro do arquivo resumo.tex.
%====================================================================
% Observação: O arquivo resumo.tex já existe dentro da 
% pasta monografia. O que o aluno precisa fazer é abrí-lo 
% e preenchê-lo com o texto do resumo. Não esqueça de 
% de colocar as palavras-chave em português. No arquivo 
% resumo.tex isso poderá ser observado. 
%====================================================================
\begin{resumo-portugues}
  \palavraschaveportugues{Interação Humano Computador (IHC), Facial Action Coding System (FACS), Usabilidade}

  \textbf{Introdução:} Este capítulo estabelece o cenário para a pesquisa, introduzindo a área de Interação Humano-Computador (IHC) e a importância da análise de emoções na experiência do usuário. Ele discute a necessidade de compreender como as emoções dos usuários influenciam a interação com sistemas de software e como essa compreensão pode melhorar a usabilidade e a experiência do usuário. O capítulo também apresenta os objetivos do trabalho, que incluem o desenvolvimento de um sistema de análise de emoções em tempo real para aplicações web, chamado "Emotion Analytics".
  \\

  \textbf{Referencial teórico:} Este capítulo fornece o embasamento teórico para a pesquisa, discutindo a relevância das emoções na IHC e os métodos existentes para a identificação de emoções. O capítulo também apresenta a metodologia FACS (Facial Action Coding System) de Paul Ekman, que permite a identificação e classificação das expressões faciais relacionadas às emoções. Além disso, discute as linguagens de programação e as tecnologias utilizadas no desenvolvimento do software.
  \\

  \textbf{Metodologia:} Este capítulo descreve a metodologia da pesquisa, que envolveu a realização de experimentos com três sites diferentes: o jogo Scary Maze, o site de notícias G1 Globo e a plataforma de vídeos Youtube. Os usuários foram instruídos a usar esses sistemas enquanto o Emotion Analytics registrava suas emoções através da webcam do computador. Os dados coletados incluíam a frequência, intensidade e duração das emoções expressas pelos usuários. O capítulo também discute o protocolo experimental e os métodos de análise dos dados.
  \\

  \textbf{Software construído:} Este capítulo detalha o desenvolvimento do sistema "Emotion Analytics". Ele discute a arquitetura do sistema, os algoritmos utilizados para a detecção de emoções, a interface do usuário e a funcionalidade do sistema, incluindo a coleta de emoções do usuário, a geração de gráficos e a refatoração dos softwares desenvolvidos. O capítulo também destaca a evolução do sistema ao longo da pesquisa, os desafios encontrados durante o desenvolvimento e como eles foram superados, e a disponibilidade do sistema como um projeto de código aberto no GitHub.
  \pagebreak

  \textbf{Resultados e análises:} Este capítulo apresenta os resultados dos experimentos realizados e a análise desses resultados. Ele discute a eficácia do sistema Emotion Analytics na detecção de emoções e fornece percepções sobre a experiência emocional dos usuários durante a utilização de software. O capítulo também discute as limitações do estudo, as possíveis fontes de erro e as implicações dos resultados para a prática e a pesquisa em IHC.
  \\

  \textbf{Conclusão:} Este capítulo resume os principais achados da pesquisa e discute suas implicações para a área de IHC. Ele destaca a contribuição do estudo para o avanço do conhecimento na áreae a potencial aplicação do sistema Emotion Analytics no desenvolvimento de interfaces mais eficientes.
\end{resumo-portugues}

%====================================================================
% Resumo em inglês: O resumo em inglês deverá ser escrito dentro 
% do arquivo abstract.tex.
%====================================================================
% Observação: O arquivo abstract.tex já existe dentro da 
% pasta monografia. O que o aluno precisa fazer é abrí-lo 
% e preenchê-lo com o texto do abstract. Não esqueça de 
% de colocar as palavras-chave em inglês. No arquivo 
% abstract.tex isso poderá ser observado. 
%====================================================================
\begin{resumo-ingles}

  \noindent Understanding users' emotions during their interaction with software systems is crucial for the development of more efficient and satisfying interfaces. Emotions play a significant role in how users perceive, react to, and engage with applications, directly influencing the user experience. In this context, this study aims to build a real-time emotion analysis system for web applications. The system, named Emotion Analytics, was developed with the purpose of capturing, identifying, and analyzing users' emotions during their interaction with applications. The Emotion Analytics system was developed using Paul Ekman's FACS (Facial Action Coding System) methodology as a basis for identifying and classifying facial expressions related to emotions. The choice of this mechanism is due to its wide use and recognition in the field, allowing a precise and objective analysis of users' emotions during their interaction with the software. Experiments conducted with the Emotion Analytics system show its effectiveness and accuracy in emotion analysis. Users interacted with different types of web applications, including a game, a news website, and a video platform, while the system recorded their emotions through the computer's webcam. The collected data provided valuable insights about the users' emotional experience, which can be used to improve the interface and usability of the tested systems. The tests carried out showed that the developed system can be a valuable tool for understanding the user experience, providing insights that can be used to improve the interface and usability of the systems.

  \palavraschaveingles{Human-Computer Interaction, Facial Action Coding System, Usability}

\end{resumo-ingles}

%====================================================================
% Comando para inclusão do sumário
%====================================================================
\tableofcontents
\thispagestyle{empty}
\cleardoublepage
%====================================================================
% Comando para inclusão da lista de figuras.
%==================================================================== 
% Observação: Caso o trabalho não tenha lista de figuras, comente 
% as 3 linhas abaixo.
%====================================================================
\pagenumbering{Roman}
\listoffigures
\cleardoublepage
%====================================================================
% Comando para inclusão da lista de tabelas.
%====================================================================
% Observação: Caso o trabalho não tenha lista de tabelas, comente 
% as 2 linhas abaixo.
%====================================================================
% \listoftables
% \cleardoublepage
%====================================================================
% Comando para inclusão da lista de abreviações e símbolos.
%====================================================================
% Observação: Caso o trabalho não tenha lista de abreviações, 
% ou lista de símbolos, comente as 2 linhas abaixo.
%====================================================================
\printnomenclature
\cleardoublepage
%====================================================================
% Capítulos da monografia.
%====================================================================
\onehalfspacing
\clearpage
\pagenumbering{arabic}
\pagestyle{cabecalhorodape}
%====================================================================
% Introdução.
%====================================================================
\chapter{Introdução}

\section{Sobre a área de atuação}

A área de Interação Humano-Computador (IHC) é um campo interdisciplinar que se concentra no estudo da interação entre seres humanos e sistemas computacionais, visando projetar e desenvolver interfaces que sejam eficientes, eficazes e proporcionem uma experiência positiva aos usuários \cite{1}.

A área de IHC é de extrema importância na atualidade, à medida que a tecnologia desempenha um papel cada vez mais central em nossa sociedade. Com a crescente dependência de sistemas computacionais em diversos aspectos de nossas vidas, desde dispositivos móveis e aplicações web até sistemas complexos de automação, é crucial garantir que essas interações sejam intuitivas, acessíveis e adequadas às necessidades dos usuários.

A interação eficaz entre humanos e computadores é essencial para melhorar a usabilidade, a produtividade e a satisfação dos usuários. Ao projetar interfaces que sejam intuitivas e fáceis de usar, podemos reduzir a curva de aprendizado e minimizar erros, proporcionando aos usuários uma experiência agradável e eficiente \cite{2}. Além disso, a área de IHC também está preocupada em considerar aspectos emocionais e sociais da interação, buscando promover engajamento, confiança e bem-estar dos usuários.

Diversos trabalhos semelhantes já foram desenvolvidos na área de IHC, abrangendo desde estudos teóricos até aplicações práticas. Entre eles, destaca-se o trabalho intitulado "O Emocard na avaliação da interação do usuário no WebGD", realizado por Angela R. B. Flores \cite{3}. Nessa pesquisa, foi proposto o uso do Emocard, uma ferramenta baseada em análise de emoções, para avaliar a interação dos usuários com o sistema WebGD.

O estudo conduzido por Flores utilizou o Emocard para capturar as emoções dos usuários durante a interação com o WebGD, um ambiente de desenvolvimento web. Por meio da análise das expressões faciais e outras reações fisiológicas dos participantes, o Emocard permitiu identificar e mensurar as emoções experimentadas pelos usuários durante a utilização do sistema.

Os resultados obtidos no trabalho mostraram que o Emocard foi eficaz na detecção de emoções, fornecendo informações valiosas sobre a experiência dos usuários com o WebGD. Com base nesses dados, foi possível compreender como as emoções influenciaram a interação e identificar aspectos do sistema que causaram respostas emocionais positivas ou negativas nos usuários.

Ao considerar o trabalho de Flores, observamos a relevância do uso de métodos e ferramentas baseados em análise de emoções para avaliar a interação humano-computador. Essa abordagem permite uma compreensão mais aprofundada das percepções e sentimentos dos usuários durante o uso de sistemas, auxiliando no aprimoramento da usabilidade e na identificação de possíveis problemas ou oportunidades de melhoria.

O estudo realizado no trabalho aponta uma abordagem semelhante a este trabalho no que se refere à utilização de emoções para análise e avaliação de software. No entanto, existem diferenças significativas entre a pesquisa estudada e esta pesquisa em relação à natureza e identificação das emoções dos usuários. No estudo anterior, as emoções relatadas pelos usuários foram classificadas como "aflição, descontentamento, depressão, sonolência, despertar, relaxamento, prazer e excitação", enquanto nesta pesquisa utilizamos as emoções "felicidade, nojo, raiva, medo, desprezo e tristeza" \cite{4}.

Além disso, os métodos de coleta das emoções também diferem entre os estudos. Na pesquisa anterior, os usuários eram solicitados a assinalar as emoções sentidas ao final da experiência de uso, enquanto nesta pesquisa as emoções são coletadas durante a própria experiência de uso por meio de algoritmos baseados na metodologia FACS de Paul Ekman \cite{5}.

Foi realizada uma extensa pesquisa bibliográfica visando encontrar estudos que representassem a experiência de uso de interfaces utilizando a metodologia FACS de Paul Ekman \cite{5}. No entanto, até o momento, não encontramos nenhum estudo que explore especificamente essas características, o que torna esta pesquisa inovadora e pioneira nesse sentido.

Essa abordagem original abre novas possibilidades para a compreensão da experiência do usuário em relação às interfaces de software, fornecendo percepções valiosas sobre as emoções e reações dos usuários durante a interação.

\section{Justificativa}

A justificativa para escrever sobre o assunto abordado neste trabalho é fundamentada na relevância e na importância que a análise e avaliação das emoções dos usuários desempenham no campo da Interface Humano-Computador (IHC). A seguir, são apresentados os principais motivos que justificam a escrita deste trabalho.

Primeiramente, a compreensão das emoções dos usuários durante a interação com sistemas de software é crucial para o desenvolvimento de interfaces mais eficientes e satisfatórias. As emoções exercem um papel significativo na forma como os usuários percebem, reagem e se envolvem com as aplicações, influenciando diretamente a experiência do usuário. Ao analisar e compreender as emoções dos usuários, é possível identificar pontos fortes e fracos das interfaces, permitindo melhorias no design e na usabilidade dos sistemas \cite{6}.

Além disso, a análise das emoções dos usuários fornece percepções valiosas sobre suas preferências, necessidades e expectativas. Ao compreender as emoções específicas evocadas pelas interfaces, os desenvolvedores podem personalizar as interações, adaptando-as às preferências individuais dos usuários. Essa abordagem personalizada contribui para a satisfação do usuário, aumentando a eficiência e a eficácia dos sistemas \cite{7}.

Outro aspecto relevante é a importância da análise das emoções dos usuários no contexto de aplicações específicas, como jogos, sites de notícias, redes sociais, entre outros. Cada tipo de aplicação possui características e objetivos distintos, e compreender as emoções dos usuários nesses contextos específicos permite desenvolver interfaces mais adequadas e atraentes. A análise das emoções dos usuários em diferentes domínios de aplicação contribui para a criação de experiências mais envolventes e satisfatórias.

Além disso, a utilização de metodologias e ferramentas computacionais para a análise das emoções apresenta uma abordagem inovadora e promissora no campo da IHC. A aplicação de algoritmos e técnicas de processamento de emoções permite coletar e interpretar grandes volumes de dados emocionais de maneira precisa e eficiente, fornecendo uma base sólida para a tomada de decisões no design e na avaliação de interfaces \cite{8}.

Por fim, a escassez de estudos relacionados à análise e avaliação das emoções dos usuários no contexto específico abordado neste trabalho também justifica a relevância deste estudo. Ao preencher essa lacuna na literatura acadêmica, este trabalho contribui para o avanço do conhecimento na área, fornecendo visões valiosas e abrindo novas perspectivas de pesquisa.

Dessa forma, considerando a importância da compreensão das emoções dos usuários na IHC, a relevância da análise em diferentes domínios de aplicação, a aplicação de metodologias computacionais e a necessidade de preencher a lacuna existente. Torna-se fundamental escrever sobre o assunto abordado neste trabalho, visando contribuir para o avanço da área e proporcionar benefícios tangíveis para o desenvolvimento de interfaces mais eficientes, personalizadas e envolventes.

\section {Objetivos}

Neste TCC, temos como objetivo principal a construção de um sistema de análise de emoções em tempo real em aplicações web. O sistema, denominado Emotion Analytics, será desenvolvido com o propósito de capturar, identificar e analisar as emoções dos usuários durante a interação com as aplicações.

O principal objetivo é fornecer aos desenvolvedores e pesquisadores uma ferramenta eficaz para compreender a experiência emocional dos usuários durante a utilização de software. Isso permitirá a identificação de possíveis problemas, melhorias e otimizações no design das aplicações, além de possibilitar uma análise mais precisa do impacto emocional das interfaces.

Os objetivos específicos deste trabalho são:

\begin{enumerate}
  \item Desenvolver um software capaz de capturar e identificar as emoções dos usuários em tempo real durante a interação com aplicações web.
  \item Utilizar a metodologia FACS (Facial Action Coding System) de Paul Ekman como base para a identificação e classificação das expressões faciais relacionadas às emoções \cite{5}.
  \item Integrar o sistema Emotion Analytics a diferentes tipos de aplicações web, permitindo a coleta e análise de dados emocionais em diversas áreas de atuação.
  \item Validar o sistema por meio de experimentos e estudos de caso, a fim de verificar sua eficácia e precisão na análise de emoções.
\end{enumerate}
  
Ao alcançar esses objetivos, espera-se que o sistema Emotion Analytics contribua significativamente para o desenvolvimento de aplicações web mais eficientes, intuitivas e emocionalmente engajadoras. Além disso, pretende-se fornecer subsídios teóricos e práticos para pesquisadores e profissionais da área, estimulando o avanço e a adoção de técnicas de análise de emoções no campo da IHC.

\section{Conteúdo}

Neste capítulo "Introdução" (Capítulo 1), foi contextualizada a pesquisa na área de Interação Humano-Computador (IHC) destacando a importância da análise de emoções na experiência do usuário. O "Referencial teórico" (Capítulo 2) fornece a base teórica, explorando a relevância das emoções na IHC e os métodos para identificação de emoções. A "Metodologia" (Capítulo 3) detalha os experimentos realizados com três sites diferentes, utilizando o sistema Emotion Analytics para registrar as emoções dos usuários. O "Software construído" (Capítulo 4) descreve o desenvolvimento do sistema Emotion Analytics, incluindo sua arquitetura, algoritmos e funcionalidades. Os "Resultados e análises" (Capítulo 5) apresentam os achados dos experimentos e discutem a eficácia do sistema na detecção de emoções. A "Conclusão" (Capítulo 6) resume os principais resultados da pesquisa e suas implicações para a área de IHC. Os "Agradecimentos" (Capítulo 7) expressam gratidão a todos que contribuíram para a realização do trabalho. Por fim, a "Bibliografia" (Capítulo 8) lista todas as referências utilizadas na pesquisa.

%====================================================================
% Referencial teórico.
%====================================================================
\chapter{Referencial teórico}

Neste capítulo, apresentaremos o referencial teórico que sustenta a pesquisa, abordando as referências bibliográficas utilizadas, as emoções e sua relevância na interação humano-computador, os mecanismos existentes para identificação de emoções, o mecanismo escolhido e as linguagens de desenvolvimento utilizadas no software.

As referências bibliográficas que fundamentam este trabalho incluem estudos relevantes na área de Interação Humano-Computador (IHC) e emoções. Dentre elas, destacam-se os trabalhos de Padovani \cite{1}, que aborda a Avaliação Ergonômica de Sistemas de Navegação em Hipertextos Fechados, Nielsen \cite{2}, com seu trabalho sobre usabilidade, e Norman \cite{6}\cite{7}, que explora os princípios do design centrado no usuário no livro "The Design of Everyday Things". Além disso, são considerados os estudos de Ekman \cite{4}\cite{5}, renomado pesquisador no campo das emoções, como "Basic Emotions" e "Emotions Revealed", e Flores \cite{3}, que propõe o uso do Emocard na avaliação da interação do usuário no WebGD.

Emoções desempenham um papel significativo na interação humano-computador, influenciando a experiência do usuário e seu engajamento com o software. Capturar as emoções durante a utilização de um sistema é fundamental para compreender o envolvimento emocional dos usuários, sua satisfação e nível de engajamento. Essa captura proporciona visões valiosas para o aprimoramento da usabilidade e da experiência do usuário, permitindo o desenvolvimento de interfaces mais adequadas e eficientes.

Existem diversos mecanismos de identificação de emoções, que variam desde abordagens tradicionais, como questionários e entrevistas, até abordagens mais tecnológicas e automatizadas. Entre os mecanismos existentes, destacam-se a análise facial computacional, a análise de expressões vocais, a detecção de padrões fisiológicos e a análise de interações comportamentais. Cada um desses mecanismos possui suas vantagens e desafios, e a escolha do mais adequado depende das características do sistema e dos objetivos da pesquisa.

Neste trabalho, optamos por utilizar o mecanismo baseado na metodologia FACS (Facial Action Coding System) desenvolvida por Paul Ekman \cite{5}. Essa metodologia utiliza algoritmos de análise facial computacional para identificar e classificar as expressões faciais associadas a diferentes emoções. A escolha desse mecanismo deve-se à sua ampla utilização e reconhecimento na área, permitindo uma análise precisa e objetiva das emoções dos usuários durante a interação com o software.

O mecanismo utilizado recebe o nome de Affectiva \cite{8} e tem como objetivo analisar imagens de rostos em larga escala, convertendo as imagens das faces coletadas em dados que apresentam as emoções desses rostos, avaliando suas micro expressões faciais \cite{4}. Esse framework foi desenvolvido por uma equipe de pesquisadores do MTI \cite{8} sendo o escolhido para a pesquisa devido à sua alta precisão e confiabilidade \cite{9}\cite{10}.

No que diz respeito às linguagens de desenvolvimento utilizadas no software, foram adotadas tecnologias e linguagens apropriadas para a implementação das funcionalidades propostas. Entre as linguagens comumente utilizadas estão HTML \cite{11}, CSS \cite{12}, JavaScript \cite{13}, Node.js \cite{14}, ReactJS \cite{15}, Redux \cite{16}, Ruby \cite{17}, entre outras. A escolha dessas linguagens baseou-se nas necessidades e requisitos específicos do projeto, visando a criação de uma aplicação robusta e interativa.

Dessa forma, o referencial teórico apresentado neste capítulo fundamenta a pesquisa. Fornecendo embasamento conceitual e acadêmico sobre as referências bibliográficas utilizadas, a importância da captura de emoções na interação humano-computador, os mecanismos de identificação de emoções existentes, o mecanismo escolhido baseado na metodologia FACS de Ekman \cite{5} e as linguagens de desenvolvimento empregadas no software.

%====================================================================
% Metodologia. 
%====================================================================
%====================================================================
% Introdução: Escreva logo após o \chapter{Introdução} a texto 
% de seu trabalho referente a introdução.
%====================================================================
\chapter{Metodologia}

\lipsum[1-2]

\nomenclature[A]{MVF}{Método dos Volumes Finitos}
\nomenclature[S]{$a$}{aceleração}

De acordo com Burden \cite{burden} e Leithold \cite{leithold}. 
Temos ainda os trabalhos de Gooch e Daniel \cite{gooch,daniel}, 
Santana \cite{santana}. Usando o \emph{gmsh} \cite{gmsh} e 
referências na internet \cite{wiki:quadrature}.

\section{Primeira Seção}

\lipsum[3-4]

\nomenclature[A]{MEF}{Método dos Elementos Finitos}
\nomenclature[S]{$\rho$}{densidade}

\section{Segunda Seção}

\lipsum[5-6]

\nomenclature[A]{MDF}{Método das Diferenças Finitas}
\nomenclature[S]{$\omega$}{frequência}
\nomenclature[A]{MDFA}{Método das Diferenças Finitas Avançado}

\section{Primeira Seção}

\lipsum[3-4]

\nomenclature[A]{MEF}{Método dos Elementos Finitos}
\nomenclature[S]{$\rho$}{densidade}

\section{Segunda Seção}

\lipsum[5-6]

\nomenclature[A]{MDF}{Método das Diferenças Finitas}
\nomenclature[S]{$\omega$}{frequência}

\section{Terceira Seção}

\lipsum{10}
%====================================================================
% Software construido.
%====================================================================
\chapter{Software construido}

\section{Caracterização do sistema}

\subsection{Sistema}

\subsection{Objetivos do sistema}

\subsection{Benefícios do Sistema}

\subsection{Escopo do sistema}

\section{Detalhamento da solução proposta}

\subsection{Home}

\subsection{Tipos de teste}

\subsection{Pessoas}

\subsection{Cadastro de usuário}

\subsection{Cadastro de tipo de teste}

\subsection{Inicio do teste}

\subsection{Relatórios}

\section{Diagramas do sistema}

\subsection{Modelo de Entidade e Relacionamento \cite{26}}

\subsection{Diagrama de casos de uso \cite{27}}

\subsection{Diagrama de classes \cite{27}}

\subsection{Diagramas de sequência \cite{28}}

\subsection{Diagramas de máquina de estado \cite{30}}

%====================================================================
% Resultados e análises.
%====================================================================
\chapter{Resultados e análises}

Neste capítulo, serão apresentados os resultados obtidos a partir dos experimentos e validação realizados com o sistema Emotion Analytics. Os dados coletados fornecem insights valiosos sobre as emoções dos usuários durante a interação com diferentes interfaces de aplicativos. Além disso, será discutida a relevância dos resultados e sua contribuição para a área de design de interfaces.

\section{Experimentos e Validação}

Os experimentos foram conduzidos seguindo a metodologia descrita no capítulo “3 Metodologia” desta monografia. Três tipos de teste foram realizados: o jogo Scary Maze \cite{18}, o site de notícias G1 Globo \cite{19} e a plataforma de vídeos YouTube \cite{20}. Cada teste teve a participação de um grupo de usuários, os quais tiveram suas emoções coletadas pelo sistema Emotion Analytics enquanto interagiam com as respectivas interfaces.

Os resultados obtidos foram analisados quanto à incidência e tempo de registro das emoções em cada teste. A seguir, são apresentados os resultados individuais para cada experimento:

\subsection{Experimento de validação com o jogo Scary Maze \cite{18}}

\begin{enumerate}
  \item A emoção "Felicidade" teve 100\% de incidência nos testes com mais de 80\% de precisão e representou 36,72\% do tempo de teste.
  \item A emoção "Nojo" teve 60\% de incidência nos testes com mais de 80\% de precisão e representou 2,25\% do tempo de teste.
  \item A emoção "Medo" não teve incidência nos testes com mais de 80\% de precisão, mas representou 1,42\% do tempo de teste em qualquer precisão.
  \item A emoção "Raiva" teve incidência em 10\% dos testes com mais de 80\% de precisão e representou 0,44\% do tempo de teste.
  \item A emoção "Desprezo" teve incidência em 20\% dos testes com mais de 80\% de precisão e representou 0,35\% do tempo de teste.
  \item A emoção "Tristeza" não teve incidência nos testes com mais de 80\% de precisão, mas representou 0,06\% do tempo de teste em qualquer precisão.
\end{enumerate}

\subsection{Experimento de validação com o site G1 Globo \cite{19}}

\begin{enumerate}
  \item A emoção "Felicidade" teve 7\% de incidência nos testes com mais de 80\% de precisão e representou 1,82\% do tempo de teste.
  \item A emoção "Nojo" teve 60\% de incidência nos testes com mais de 80\% de precisão e representou 3,26\% do tempo de teste.
  \item A emoção "Medo" teve 55\% de incidência nos testes com mais de 80\% de precisão e representou 4,32\% do tempo de teste em qualquer precisão.
  \item A emoção "Raiva" teve incidência em 10\% dos testes com mais de 80\% de precisão e representou 20,35\% do tempo de teste.
  \item A emoção "Desprezo" teve incidência em 15\% dos testes com mais de 80\% de precisão e representou 2,89\% do tempo de teste.
  \item A emoção "Tristeza" teve 57\% de incidência nos testes com mais de 80\% de precisão e representou 2,18\% do tempo de teste.
\end{enumerate}

\subsection{Experimento de validação com o site YouTube \cite{20}}

\begin{enumerate}
  \item A emoção "Felicidade" teve 77\% de incidência nos testes com mais de 80\% de precisão e representou 25,54\% do tempo de teste.
  \item A emoção "Nojo" teve 14\% de incidência nos testes com mais de 80\% de precisão e representou 1,56\% do tempo de teste.
  \item A emoção "Medo" não teve incidência nos testes com mais de 80\% de precisão, mas representou 1,84\% do tempo de teste em qualquer precisão.
  \item A emoção "Raiva" teve incidência em 5\% dos testes com mais de 80\% de precisão e representou 0,87\% do tempo de teste.
  \item A emoção "Desprezo" teve incidência em 21\% dos testes com mais de 80\% de precisão e representou 7,16\% do tempo de teste.
  \item A emoção "Tristeza" não teve incidência nos testes com mais de 80\% de precisão, mas representou 0,03\% do tempo de teste em qualquer precisão.
\end{enumerate}

\section{Análise dos Resultados}

Os resultados dos experimentos revelam informações importantes sobre as emoções dos usuários durante a interação com as diferentes interfaces. Cada teste apresentou uma predominância de uma emoção específica, indicando tendências e traços da experiência de uso das aplicações.

No experimento com o jogo Scary Maze \cite{18}, foi observada uma alta incidência e tempo da emoção "Felicidade", sugerindo que os colaboradores se divertiram com o jogo. Além disso, a incidência da emoção "Nojo" pode estar relacionada à imagem do monstro apresentada ao final do percurso no jogo.

No experimento com o site de notícias G1 Globo \cite{19}, a predominância da emoção "Raiva" em 82\% dos casos com mais de 80\% de precisão e em 20,35\% do tempo de teste, juntamente com as altas incidências das emoções "Nojo", "Medo" e "Tristeza", indica que o conteúdo das notícias apresentadas na tela inicial do site gerou sentimentos negativos e possivelmente desconfortantes nos usuários.

No experimento com o site YouTube \cite{20}, a predominância da emoção "Felicidade" em 77\% dos casos com mais de 80\% de precisão e em 25,54\% do tempo de teste sugere que os usuários ficaram satisfeitos ao assistir ao vídeo selecionado. Esse resultado pode estar relacionado ao fato de que eles estavam logados em suas contas, o que proporcionou a exibição de vídeos relacionados às suas preferências individuais.

Essas conclusões baseadas nos dados obtidos nos experimentos contribuem para uma melhor compreensão das emoções dos usuários durante a interação com interfaces de aplicativos.

\section{Publicação do Artigo e Código Fonte Aberto}

Cabe ressaltar que os resultados obtidos neste trabalho foram apresentados no "21º Simpósio de Iniciação Científica e Tecnológica da Faculdade de Tecnologia de São Paulo" \cite{25}. A publicação desse artigo científico reforça a credibilidade e relevância dos resultados encontrados, contribuindo para a disseminação do conhecimento na área.

Além disso, é importante destacar que o código fonte do sistema Emotion Analytics está disponível de forma aberta em um repositório do GitHub \cite{22}\cite{23}. Essa abertura possibilita que outros pesquisadores e desenvolvedores tenham acesso ao código e possam contribuir para sua evolução e aprimoramento, fortalecendo a colaboração e o avanço na área de análise de emoções em interfaces de aplicativos.

%====================================================================
% Conclusão.
%====================================================================
\input{capitulos/6-conclusão.tex}
%====================================================================
% Referências bibliográficas
%====================================================================
\newpage
\def\thispagestyle#1{}
%\bibliographystyle{babplain-fl}
\bibliographystyle{bababbrv-lf}
\bibliography{bibliografia}
%====================================================================
% Apêndices da monografia
% Pode ser que o trabalho tenha vários apêndices. Coloque os 
% apêndices em uma pasta e dentro da pasta crie o arquivo tex com o 
% texto associado ao apêndice. 
%====================================================================
%\begin{appendices}
%%====================================================================
% Texto referente ao primeiro apêndice  
%====================================================================
\chapter{Apêndice}




%%====================================================================
% Texto referente ao segunda apêndice  
%====================================================================
\chapter{Apêndice}

%\end{appendices}

\end{document}
