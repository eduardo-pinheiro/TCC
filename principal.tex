%====================================================================
% Universidade Federal de Uberlândia
% Faculdade de Matemática
% Curso: Bacharelado em Estatística
% Modelo para elaboração de monografia referentes aos trabalhos de
% conclusão de curso do Bacharelado em Estatística
% Autor: Prof. Dr. Alessandro Alves Santana
%====================================================================
\documentclass[12pt,openright]{book}
%====================================================================
% Pacote de estilo que tem por finalidade carregar pacotes do LaTeX
% bem como definir configurações do texto da monografia. 
%====================================================================
\usepackage{estilo}
%====================================================================
% Inserção do arquivo com a lista de abreviaturas e símbolos.
%====================================================================
\makenomenclature
%====================================================================
% Documento Principal
%====================================================================
\begin{document}
\tolerance=5000
%====================================================================
% Para que as linhas do texto sejam numeradas com a finalidade 
% de facilitar a correção pelo orientador, basta descomentar o
% o comando abaixo. Esse comando faz com que o arquivo pdf gerado 
% após a compilação sem erros tenham as linhas da monografia sejam
% enumeradas. Com as linhas numeradas o orientador poderá localizar 
% e descrever melhor para os orientandos os locais do texto que 
% exigem correção.
%====================================================================
%\linenumbers
%====================================================================
% O comando \capa tem 6 argumentos de entrada os quais são 
% descritos abaixo. Tem por função gerar a capa da monografia. 
%
% \capa
% {título da monografia}
% {nome do orientando}
% {ano de defesa da monografia}
% {nome do orientador}
% {nome do segundo membro da banca}
% {nome do terceiro membro da banca}
%
% Observações: 
%
%  1- Escrever o título do trabalho em letras maíusculas.
%       Exemplo: ANÁLISE DE REGRESSÃO EM PROBLEMAS DE FÍSICA
%  2- Nome do aluno com iniciais em letras maiúsculas.
%       Exemplo: Alberto Barbosa da Cunha
%  3- Ano da defesa: 
%       Exemplo: 2018
%  4- Nome do Orientador e demais membros da banca de defesa: 
%     Escrever o nome apenas com as iniciais em letras maiúsculas 
%     como no nome do aluno.
%       Exemplo: João Carlos Silva de Almeida    
%====================================================================
\capa
{SISTEMA PARA AVALIAÇÃO DE INTERFACES A
PARTIR DAS EMOÇÕES DOS USUÁRIOS}
{Eduardo Felipe Santos Pinheiro}
{2023}
{Marcelo Duduchi Feitosa}
{Nome completo do segundo membro da banca de defesa}
{Nome completo do terceiro membro da banca de defesa}
%====================================================================
% Agradecimentos: Os agradecimentos deverão ser escritos dentro do 
% arquivo agradecimentos.tex. 
%====================================================================     
% Observação: O arquivo agradecimentos.tex já existe dentro da 
% pasta monografia. O que o aluno precisa fazer é abrí-lo 
% e preenchê-lo com o texto dos agradecimentos.  
%====================================================================
%====================================================================
% Agradecimentos: Escreva no ambiente abaixo os agradecimentos.
%  
%  Exemplo:
%  
%  \begin{agradecimentos}
%     Agradeço a toda minha família e a todos os meus amigos 
%     pelo apoio nessa importante fase do meu processo de formação
%     profissional.
%  \end{agradecimentos}
%====================================================================

\begin{agradecimentos}
  Primeiramente, gostaria de expressar minha profunda gratidão ao CNPQ pela oportunidade de participar do Programa Institucional de Bolsas de Iniciação Científica (PBIC). A experiência e o conhecimento adquiridos durante a minha participação no programa foram fundamentais para a realização deste trabalho. Agradeço pela confiança depositada em mim e pelo suporte contínuo que me permitiu desenvolver e aprimorar minhas habilidades de pesquisa.
  \\

  Em segundo lugar, gostaria de agradecer ao meu orientador, o Professor Marcelo Duduchi. Sua orientação, paciência e conhecimento foram inestimáveis para a realização deste trabalho. Agradeço por ter me desafiado a pensar criticamente, a questionar e a buscar soluções inovadoras para os desafios encontrados durante o desenvolvimento do projeto. Sua orientação e apoio foram fundamentais para o sucesso deste trabalho.
  \\

  Por fim, gostaria de expressar minha gratidão à Faculdade de Tecnologia de São Paulo (FATEC-SP). Agradeço a todos os professores e colegas que contribuíram para a minha formação acadêmica e pessoal. A experiência adquirida durante o curso foi fundamental para a realização deste trabalho e para o meu desenvolvimento como profissional na área de tecnologia da informação.
  \\
  
  Este trabalho é o resultado de muitas horas de estudo, pesquisa e dedicação, e eu sou grato por todas as pessoas e instituições que me apoiaram ao longo desta jornada. Agradeço a todos que contribuíram, direta ou indiretamente, para a realização deste trabalho.
\end{agradecimentos}

%====================================================================
% Resumo em português: O resumo em português deverá ser escrito 
% dentro do arquivo resumo.tex.
%====================================================================
% Observação: O arquivo resumo.tex já existe dentro da 
% pasta monografia. O que o aluno precisa fazer é abrí-lo 
% e preenchê-lo com o texto do resumo. Não esqueça de 
% de colocar as palavras-chave em português. No arquivo 
% resumo.tex isso poderá ser observado. 
%====================================================================
%====================================================================
% Resumo em português: Escreva no ambiente abaixo o resumo colocando 
% no final, via comando \palavraschaveportugues, as palavras-chave
% em português. Não coloque ponto final no final das palavras-chave
% pois o comando já insere esse ponto.
%  
%  Exemplo:
%  
%  \begin{resumo-portugues}
%     O presente trabalho foi desenvolvido com a finalidade de 
%     realizar um estudo sobre diversas metodologias de análise 
%     de dados.
%
%     \palavraschaveportugues{Análise de dados, inferência}
%  \end{resumo-portugues}
%====================================================================

\begin{resumo-portugues}
  \palavraschaveportugues{Interação Humano Computador (IHC), Facial Action Coding System (FACS), Usabilidade}

  \textbf{Introdução:} Este capítulo estabelece o cenário para a pesquisa, introduzindo a área de Interação Humano-Computador (IHC) e a importância da análise de emoções na experiência do usuário. Ele discute a necessidade de compreender como as emoções dos usuários influenciam a interação com sistemas de software e como essa compreensão pode melhorar a usabilidade e a experiência do usuário. O capítulo também apresenta os objetivos do trabalho, que incluem o desenvolvimento de um sistema de análise de emoções em tempo real para aplicações web, chamado "Emotion Analytics".
  \\

  \textbf{Referencial teórico:} Este capítulo fornece o embasamento teórico para a pesquisa, discutindo a relevância das emoções na IHC e os métodos existentes para a identificação de emoções. Ele explora a teoria das emoções, a psicologia cognitiva e a neurociência para entender como as emoções são expressas e percebidas. O capítulo também apresenta a metodologia FACS (Facial Action Coding System) de Paul Ekman, que permite a identificação e classificação das expressões faciais relacionadas às emoções. Além disso, discute as linguagens de programação e as tecnologias utilizadas no desenvolvimento do software.
  \\

  \textbf{Metodologia:} Este capítulo descreve a metodologia da pesquisa, que envolveu a realização de experimentos com três sites diferentes: o jogo Scary Maze, o site de notícias G1 Globo e a plataforma de vídeos Youtube. Os usuários foram instruídos a usar esses sistemas enquanto o Emotion Analytics registrava suas emoções através da webcam do computador. Os dados coletados incluíam a frequência, intensidade e duração das emoções expressas pelos usuários. O capítulo também discute os critérios de seleção dos participantes, o protocolo experimental e os métodos de análise dos dados.
  \\

  \textbf{Software construído:} Este capítulo detalha o desenvolvimento do sistema "Emotion Analytics". Ele discute a arquitetura do sistema, os algoritmos utilizados para a detecção de emoções, a interface do usuário e a funcionalidade do sistema, incluindo a coleta de emoções do usuário, a geração de gráficos e a refatoração dos softwares desenvolvidos. O capítulo também destaca a evolução do sistema ao longo da pesquisa, os desafios encontrados durante o desenvolvimento e como eles foram superados, e a disponibilidade do sistema como um projeto de código aberto no GitHub.
  \pagebreak

  \textbf{Resultados e análises:} Este capítulo apresenta os resultados dos experimentos realizados e a análise desses resultados. Ele discute a eficácia do sistema Emotion Analytics na detecção de emoções e fornece percepções sobre a experiência emocional dos usuários durante a utilização de software. O capítulo também discute as limitações do estudo, as possíveis fontes de erro e as implicações dos resultados para a prática e a pesquisa em IHC.
  \\

  \textbf{Conclusão:} Este capítulo resume os principais achados da pesquisa e discute suas implicações para a área de IHC. Ele destaca a contribuição do estudo para o avanço do conhecimento na áreae a potencial aplicação do sistema Emotion Analytics no desenvolvimento de interfaces mais eficientes e emocionalmente engajadoras. O capítulo também discute as limitações do estudo e sugere direções para pesquisas futuras.
\end{resumo-portugues}

%====================================================================
% Resumo em inglês: O resumo em inglês deverá ser escrito dentro 
% do arquivo abstract.tex.
%====================================================================
% Observação: O arquivo abstract.tex já existe dentro da 
% pasta monografia. O que o aluno precisa fazer é abrí-lo 
% e preenchê-lo com o texto do abstract. Não esqueça de 
% de colocar as palavras-chave em inglês. No arquivo 
% abstract.tex isso poderá ser observado. 
%====================================================================
%====================================================================
% Resumo em inglês: Escreva no ambiente abaixo o resumo colocando 
% no final, via comando \palavraschaveingles, as palavras-chave
% em inglês. Não coloque ponto final no final das palavras-chave
% pois o comando já insere esse ponto.
%  
%  Exemplo:
%  
%  \begin{resumo-ingles}
%     Many engineering problems involve the analysis of the variation 
%     rate, i.e., the analysis of one or more physical derived 
%     properties over time and/or space.
%
%     \palavraschaveingles{Finite Volume Method, High Order Method}
%  \end{resumo-ingles}
%====================================================================

\begin{resumo-ingles}
  \palavraschaveingles{Human Computer Interaction (HCI), Facial Action Coding System (FACS), Usability}

  \textbf{Introduction:} This chapter sets the stage for the research by introducing the field of Human-Computer Interaction (HCI) and the importance of emotion analysis in the user experience. It discusses the need to understand how user emotions influence the interaction with software systems and how this understanding can improve usability and user experience. The chapter also presents the objectives of the work, which include the development of a real-time emotion analysis system for web applications called "Emotion Analytics."
  \\

  \textbf{Theoretical Framework:} This chapter provides the theoretical foundation for the research by discussing the relevance of emotions in HCI and the existing methods for emotion identification. It explores the theory of emotions, cognitive psychology, and neuroscience to understand how emotions are expressed and perceived. The chapter also introduces Paul Ekman's Facial Action Coding System (FACS), which allows for the identification and classification of facial expressions related to emotions. Additionally, it discusses the programming languages and technologies used in the software development.
  \\

  \textbf{Methodology:} This chapter describes the research methodology, which involved conducting experiments with three different websites: the Scary Maze game, the G1 Globo news website, and the YouTube video platform. Users were instructed to use these systems while Emotion Analytics recorded their emotions through the computer webcam. The collected data included the frequency, intensity, and duration of emotions expressed by the users. The chapter also discusses the participant selection criteria, the experimental protocol, and the data analysis methods.
  \\

  \textbf{Software Development:} This chapter details the development of the "Emotion Analytics" system. It discusses the system architecture, the algorithms used for emotion detection, the user interface, and the system's functionality, including emotion collection from the user, graph generation, and software refactoring. The chapter also highlights the evolution of the system throughout the research, the challenges encountered during development and how they were overcome, and the availability of the system as an open-source project on GitHub.
  \pagebreak

  \textbf{Results and Analysis:} This chapter presents the results of the conducted experiments and their analysis. It discusses the effectiveness of the Emotion Analytics system in emotion detection and provides insights into the users' emotional experience during software usage. The chapter also discusses the study's limitations, potential sources of error, and the implications of the results for practice and research in HCI.
  \\

  \textbf{Conclusion:} This chapter summarizes the key findings of the research and discusses their implications for the field of HCI. It emphasizes the study's contribution to advancing knowledge in the field and the potential application of the Emotion Analytics system in the development of more efficient and emotionally engaging interfaces. The chapter also discusses the limitations of the study and suggests directions for future research.
\end{resumo-ingles}

%====================================================================
% Comando para inclusão do sumário
%====================================================================
\tableofcontents
\thispagestyle{empty}
\cleardoublepage
%====================================================================
% Comando para inclusão da lista de figuras.
%==================================================================== 
% Observação: Caso o trabalho não tenha lista de figuras, comente 
% as 3 linhas abaixo.
%====================================================================
\pagenumbering{Roman}
\listoffigures
\cleardoublepage
%====================================================================
% Comando para inclusão da lista de tabelas.
%====================================================================
% Observação: Caso o trabalho não tenha lista de tabelas, comente 
% as 2 linhas abaixo.
%====================================================================
% \listoftables
% \cleardoublepage
%====================================================================
% Comando para inclusão da lista de abreviações e símbolos.
%====================================================================
% Observação: Caso o trabalho não tenha lista de abreviações, 
% ou lista de símbolos, comente as 2 linhas abaixo.
%====================================================================
\printnomenclature
\cleardoublepage
%====================================================================
% Capítulos da monografia.
%====================================================================
\onehalfspacing
\clearpage
\pagenumbering{arabic}
\pagestyle{cabecalhorodape}
%====================================================================
% Introdução.
%====================================================================
%====================================================================
% Introdução: Escreva logo após o \chapter{Introdução} a texto 
% de seu trabalho referente a introdução.
%====================================================================
\chapter{Introdução}

\lipsum[1-2]

\nomenclature[A]{MVF}{Método dos Volumes Finitos}
\nomenclature[S]{$a$}{aceleração}

De acordo com Burden \cite{burden} e Leithold \cite{leithold}. 
Temos ainda os trabalhos de Gooch e Daniel \cite{gooch,daniel}, 
Santana \cite{santana}. Usando o \emph{gmsh} \cite{gmsh} e 
referências na internet \cite{wiki:quadrature}.

\section{Primeira Seção}

\lipsum[3-4]

\nomenclature[A]{MEF}{Método dos Elementos Finitos}
\nomenclature[S]{$\rho$}{densidade}

\section{Segunda Seção}

\lipsum[5-6]

\nomenclature[A]{MDF}{Método das Diferenças Finitas}
\nomenclature[S]{$\omega$}{frequência}
\nomenclature[A]{MDFA}{Método das Diferenças Finitas Avançado}

\section{Primeira Seção}

\lipsum[3-4]

\nomenclature[A]{MEF}{Método dos Elementos Finitos}
\nomenclature[S]{$\rho$}{densidade}

\section{Segunda Seção}

\lipsum[5-6]

\nomenclature[A]{MDF}{Método das Diferenças Finitas}
\nomenclature[S]{$\omega$}{frequência}

\section{Terceira Seção}

\lipsum{10}
%====================================================================
% Desenvolvimento.
%====================================================================
%====================================================================
% Desenvolvimento: Escreva logo após o \chapter{Desenvolvimento} a 
% texto de seu trabalho referente ao desenvolvimento. 
%====================================================================
\chapter{Desenvolvimento}





%====================================================================
% Fundamentação Teórica. 
%====================================================================
%====================================================================
% Fundamentação teórica: Escreva logo após o 
% \chapter{Fundamentação Teórica} a texto de seu trabalho referente 
% aos fundamentos teóricos do que desenvolveu em seu estudo. 
%====================================================================
\chapter{Fundamentação Teórica}



%====================================================================
% Metodologia.
%====================================================================
%====================================================================
% Metodologia: Escreva logo após o 
% \chapter{Metodologia} o texto de seu trabalho referente 
% aos métodos utilizados no que desenvolvimento de seu estudo. 
%====================================================================
\chapter{Metodologia}



%====================================================================
% Resultados.
%====================================================================
%====================================================================
% Resultados: Escreva logo após o 
% \chapter{Resultados} o texto de seu trabalho referente 
% aos resultados obtidos no desenvolvimento de seu estudo. 
%====================================================================
\chapter{Resultados}


%====================================================================
% Conclusões.
%====================================================================
%====================================================================
% Conclusões: Escreva logo após o 
% \chapter{Conclusões} o texto de seu trabalho referente 
% as conclusões de seu estudo. 
%====================================================================
\chapter{Conclusões}



%====================================================================
% Referências bibliográficas
%====================================================================
\newpage
\def\thispagestyle#1{}
%\bibliographystyle{babplain-fl}
\bibliographystyle{bababbrv-lf}
\bibliography{bibliografia}
%====================================================================
% Apêndices da monografia
% Pode ser que o trabalho tenha vários apêndices. Coloque os 
% apêndices em uma pasta e dentro da pasta crie o arquivo tex com o 
% texto associado ao apêndice. 
%====================================================================
%\begin{appendices}
%%====================================================================
% Texto referente ao primeiro apêndice  
%====================================================================
\chapter{Apêndice}




%%====================================================================
% Texto referente ao segunda apêndice  
%====================================================================
\chapter{Apêndice}

%\end{appendices}

\end{document}
