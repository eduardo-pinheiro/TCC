\begin{resumo-ingles}
  \palavraschaveingles{Human-Computer Interaction (HCI), Facial Action Coding System (FACS), Usability}

  \textbf{Introduction:} This chapter sets the scene for the research, introducing the field of Human-Computer Interaction (HCI) and the importance of emotion analysis in user experience. It discusses the need to understand how user emotions influence interaction with software systems and how this understanding can improve usability and user experience. The chapter also presents the objectives of the work, which include the development of a real-time emotion analysis system for web applications, called "Emotion Analytics".
  \\

  \textbf{Theoretical framework:} This chapter provides the theoretical grounding for the research, discussing the relevance of emotions in HCI and the existing methods for emotion identification. The chapter also introduces Paul Ekman's Facial Action Coding System (FACS) methodology, which allows for the identification and classification of facial expressions related to emotions. Moreover, it discusses the programming languages and technologies used in the software development.
  \\

  \textbf{Methodology:} This chapter describes the research methodology, which involved conducting experiments with three different websites: the Scary Maze game, G1 Globo news website, and the Youtube video platform. Users were instructed to use these systems while Emotion Analytics recorded their emotions via the computer's webcam. The collected data included the frequency, intensity, and duration of the emotions expressed by the users. The chapter also discusses the experimental protocol and the methods of data analysis.
  \\

  \textbf{Software development:} This chapter details the development of the "Emotion Analytics" system. It discusses the system's architecture, the algorithms used for emotion detection, the user interface, and the system functionality, including the collection of user emotions, graph generation, and refactoring of the developed software. The chapter also highlights the system's evolution throughout the research, the challenges encountered during the development and how they were overcome, and the system's availability as an open-source project on GitHub.
  \pagebreak

  \textbf{Results and analyses:} This chapter presents the results of the conducted experiments and the analysis of these results. It discusses the effectiveness of the Emotion Analytics system in detecting emotions and provides insights into users' emotional experience while using the software. The chapter also discusses the study's limitations, the possible sources of error, and the implications of the results for practice and research in HCI.
  \\

  \textbf{Conclusion:} This chapter summarizes the key findings of the research and discusses their implications for the HCI field. It highlights the study's contribution to the advancement of knowledge in the area and the potential application of the Emotion Analytics system in the development of more efficient interfaces.
\end{resumo-ingles}
