%====================================================================
% Resumo em inglês: Escreva no ambiente abaixo o resumo colocando 
% no final, via comando \palavraschaveingles, as palavras-chave
% em inglês. Não coloque ponto final no final das palavras-chave
% pois o comando já insere esse ponto.
%  
%  Exemplo:
%  
%  \begin{resumo-ingles}
%     Many engineering problems involve the analysis of the variation 
%     rate, i.e., the analysis of one or more physical derived 
%     properties over time and/or space.
%
%     \palavraschaveingles{Finite Volume Method, High Order Method}
%  \end{resumo-ingles}
%====================================================================

\begin{resumo-ingles}
  \palavraschaveingles{Human Computer Interaction (HCI), Facial Action Coding System (FACS), Usability}

  \textbf{Introduction:} This chapter sets the stage for the research by introducing the field of Human-Computer Interaction (HCI) and the importance of emotion analysis in the user experience. It discusses the need to understand how user emotions influence the interaction with software systems and how this understanding can improve usability and user experience. The chapter also presents the objectives of the work, which include the development of a real-time emotion analysis system for web applications called "Emotion Analytics."
  \\

  \textbf{Theoretical Framework:} This chapter provides the theoretical foundation for the research by discussing the relevance of emotions in HCI and the existing methods for emotion identification. It explores the theory of emotions, cognitive psychology, and neuroscience to understand how emotions are expressed and perceived. The chapter also introduces Paul Ekman's Facial Action Coding System (FACS), which allows for the identification and classification of facial expressions related to emotions. Additionally, it discusses the programming languages and technologies used in the software development.
  \\

  \textbf{Methodology:} This chapter describes the research methodology, which involved conducting experiments with three different websites: the Scary Maze game, the G1 Globo news website, and the YouTube video platform. Users were instructed to use these systems while Emotion Analytics recorded their emotions through the computer webcam. The collected data included the frequency, intensity, and duration of emotions expressed by the users. The chapter also discusses the participant selection criteria, the experimental protocol, and the data analysis methods.
  \\

  \textbf{Software Development:} This chapter details the development of the "Emotion Analytics" system. It discusses the system architecture, the algorithms used for emotion detection, the user interface, and the system's functionality, including emotion collection from the user, graph generation, and software refactoring. The chapter also highlights the evolution of the system throughout the research, the challenges encountered during development and how they were overcome, and the availability of the system as an open-source project on GitHub.
  \pagebreak

  \textbf{Results and Analysis:} This chapter presents the results of the conducted experiments and their analysis. It discusses the effectiveness of the Emotion Analytics system in emotion detection and provides insights into the users' emotional experience during software usage. The chapter also discusses the study's limitations, potential sources of error, and the implications of the results for practice and research in HCI.
  \\

  \textbf{Conclusion:} This chapter summarizes the key findings of the research and discusses their implications for the field of HCI. It emphasizes the study's contribution to advancing knowledge in the field and the potential application of the Emotion Analytics system in the development of more efficient and emotionally engaging interfaces. The chapter also discusses the limitations of the study and suggests directions for future research.
\end{resumo-ingles}
