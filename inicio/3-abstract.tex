\begin{resumo-ingles}

  Understanding users' emotions during their interaction with software systems is crucial for the development of more efficient and satisfying interfaces. Emotions play a significant role in how users perceive, react to, and engage with applications, directly influencing the user experience. In this context, this study aims to build a real-time emotion analysis system for web applications. The system, named Emotion Analytics, was developed with the purpose of capturing, identifying, and analyzing users' emotions during their interaction with applications. The Emotion Analytics system was developed using Paul Ekman's FACS (Facial Action Coding System) methodology as a basis for identifying and classifying facial expressions related to emotions. The choice of this mechanism is due to its wide use and recognition in the field, allowing a precise and objective analysis of users' emotions during their interaction with the software. Experiments conducted with the Emotion Analytics system demonstrated its effectiveness and accuracy in emotion analysis. Users interacted with different types of web applications, including a game, a news website, and a video platform, while the system recorded their emotions through the computer's webcam. The collected data provided valuable insights about the users' emotional experience, which can be used to improve the interface and usability of the tested systems. The tests carried out showed that the developed system can be a valuable tool for understanding the user experience, providing insights that can be used to improve the interface and usability of the systems.

  \palavraschaveingles{Human-Computer Interaction, Facial Action Coding System, Usability}

\end{resumo-ingles}
