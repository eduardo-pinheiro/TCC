\begin{resumo-portugues}
  
  \noindent A compreensão das emoções dos usuários durante a interação com sistemas de software é crucial para o desenvolvimento de interfaces mais eficientes e satisfatórias. As emoções exercem um papel significativo na forma como os usuários percebem, reagem e se envolvem com as aplicações, influenciando diretamente a experiência do usuário. Nesse contexto, o presente trabalho tem como objetivo a construção de um sistema de análise de emoções em tempo real em aplicações web. O sistema, denominado Emotion Analytics, foi desenvolvido com o propósito de capturar, identificar e analisar as emoções dos usuários durante a interação com as aplicações. O sistema Emotion Analytics foi desenvolvido utilizando a metodologia FACS (Facial Action Coding System) de Paul Ekman como base para a identificação e classificação das expressões faciais relacionadas às emoções. A escolha desse mecanismo deve-se à sua ampla utilização e reconhecimento na área, permitindo uma análise precisa e objetiva das emoções dos usuários durante a interação com o software. Os experimentos realizados com o sistema Emotion Analytics mostram sua eficácia e precisão na análise de emoções. Os usuários interagiram com diferentes tipos de aplicações web, incluindo um jogo, um site de notícias e uma plataforma de vídeos, enquanto o sistema registrava suas emoções através da webcam do computador. Os dados coletados forneceram insights valiosos sobre a experiência emocional dos usuários, que podem ser usados para melhorar a interface e a usabilidade dos sistemas testados. Os testes realizados mostraram que o sistema desenvolvido pode ser uma ferramenta valiosa para a compreensão da experiência do usuário, fornecendo percepções que podem ser usados para melhorar a interface e a usabilidade dos sistemas.

  \palavraschaveportugues{Interação Humano Computador, Facial Action Coding System, Usabilidade}

\end{resumo-portugues}
