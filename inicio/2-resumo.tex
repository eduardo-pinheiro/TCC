\begin{resumo-portugues}
  \palavraschaveportugues{Interação Humano Computador (IHC), Facial Action Coding System (FACS), Usabilidade}

  \textbf{Introdução:} Este capítulo estabelece o cenário para a pesquisa, introduzindo a área de Interação Humano-Computador (IHC) e a importância da análise de emoções na experiência do usuário. Ele discute a necessidade de compreender como as emoções dos usuários influenciam a interação com sistemas de software e como essa compreensão pode melhorar a usabilidade e a experiência do usuário. O capítulo também apresenta os objetivos do trabalho, que incluem o desenvolvimento de um sistema de análise de emoções em tempo real para aplicações web, chamado "Emotion Analytics".
  \\

  \textbf{Referencial teórico:} Este capítulo fornece o embasamento teórico para a pesquisa, discutindo a relevância das emoções na IHC e os métodos existentes para a identificação de emoções. O capítulo também apresenta a metodologia FACS (Facial Action Coding System) de Paul Ekman, que permite a identificação e classificação das expressões faciais relacionadas às emoções. Além disso, discute as linguagens de programação e as tecnologias utilizadas no desenvolvimento do software.
  \\

  \textbf{Metodologia:} Este capítulo descreve a metodologia da pesquisa, que envolveu a realização de experimentos com três sites diferentes: o jogo Scary Maze, o site de notícias G1 Globo e a plataforma de vídeos Youtube. Os usuários foram instruídos a usar esses sistemas enquanto o Emotion Analytics registrava suas emoções através da webcam do computador. Os dados coletados incluíam a frequência, intensidade e duração das emoções expressas pelos usuários. O capítulo também discute o protocolo experimental e os métodos de análise dos dados.
  \\

  \textbf{Software construído:} Este capítulo detalha o desenvolvimento do sistema "Emotion Analytics". Ele discute a arquitetura do sistema, os algoritmos utilizados para a detecção de emoções, a interface do usuário e a funcionalidade do sistema, incluindo a coleta de emoções do usuário, a geração de gráficos e a refatoração dos softwares desenvolvidos. O capítulo também destaca a evolução do sistema ao longo da pesquisa, os desafios encontrados durante o desenvolvimento e como eles foram superados, e a disponibilidade do sistema como um projeto de código aberto no GitHub.
  \pagebreak

  \textbf{Resultados e análises:} Este capítulo apresenta os resultados dos experimentos realizados e a análise desses resultados. Ele discute a eficácia do sistema Emotion Analytics na detecção de emoções e fornece percepções sobre a experiência emocional dos usuários durante a utilização de software. O capítulo também discute as limitações do estudo, as possíveis fontes de erro e as implicações dos resultados para a prática e a pesquisa em IHC.
  \\

  \textbf{Conclusão:} Este capítulo resume os principais achados da pesquisa e discute suas implicações para a área de IHC. Ele destaca a contribuição do estudo para o avanço do conhecimento na áreae a potencial aplicação do sistema Emotion Analytics no desenvolvimento de interfaces mais eficientes.
\end{resumo-portugues}
